
\documentclass{beamer}
\usepackage{Styles/style_pres}
% This packages brings is all presentation specfic custom commands and automatically sets the Princeton-inspired presentation style.

\title{Title}
\author{Names}
\date{Date}

\AllowAnimationsfalse
% This command does not allow animations of any kind to take place.
% The default is set to "\AllowAnimationstrue".

\begin{document}

\titleandoutline
% This command generates the title and outline slides.

\bettersection{Motivation and Background}
% This command reates a repeated outline slide and ensures name of section is on every subslides subtitle.

\betterframe{Motivation Slide}{
    % This command creates a frame with an empty starting slide for slide animation.
    % If you would not like a empty starting frame the run the command "\betterframe[false]{Motivation}".
    \bo{enumerate}{0.25cm}
    % This command animates enumerate (or itemize) with 0.25cm spacing between bullets.
        \bitem{Applied Math Path:}
        \bi{itemize}{0.125cm}
        % This command animates itemize (or enumerate) with 0.5cm spacing between bullets and the parent "\bo{#1}{#2}\eo{#1}".
            \item Motivate (\textit{motivation}) why the your area of research is important to the real world and (\textit{problem}) why the current mathematical tools are inadequate at solving related problems.
            % This command is equivalent to \item \textbf{#1} = \bitem{#1}.
        \ei{itemize}
        \bitem{Pure Math Path:}
        \bi{itemize}{0.125cm}
            \item Motivate (\textit{motivation}) why the proposed area of mathematical research has the possibility to expand human understanding and (\textit{problem}) why related problems are difficult.
        \ei{itemize}
        \bitem{Experimental Path:}
        \bi{itemize}{0.125cm}
            \item Motivate (\textit{motivation}) why the proposed experimental results have the possibility to expand human understanding and (\textit{problem}) why these experimental results are difficult to gather.
        \ei{itemize}
    \eo{enumerate}
}

\betterframe{Background Slide}{
    \bo{enumerate}{0.25cm}
        \iitem{Applied Math Path:}
        \bi{itemize}{0.125cm}
            \item Review relevant literature (\textit{literature review}) attempting to solve the same problem.
            Next, (\textit{problem focus}) discuss what aspect of the problem you plan to focus on during this talk.
            % This command is equivalent to \item \textit{#1} = \bitem{#1}.
            Finally, (\textit{proposed solution}) describe the proposed solution in one \alert{nontechnical} sentence.
        \ei{itemize}
        \iitem{Pure Math Path:}
        \bi{itemize}{0.125cm}
            \item Review relevant literature (\textit{literature review}) of (1) investigative studies that substantiate your existence argument for a knowledge gap and/or (2) previous attempts to solve the same problem.
            Next, (\textit{problem focus}) discuss what aspect of the problem you plan to focus on during this talk.
            Finally, (\textit{proposed solution}) describe the proposed solution in one \alert{nontechnical} sentence.
        \ei{itemize}
        \iitem{Experimental Path:}
        \bi{itemize}{0.125cm}
            \item See Pure Math Path.
        \ei{itemize}
    \eo{enumerate}
}

\bettersection{Set Up}

\betterframe{Set Up Slide}{
    \bo{enumerate}{0.25cm}
        \bitem{Applied Math Path:}
        \bo{itemize}{0.125cm}
            \item Set up (\textit{set up}) the mathematical prerequisite knowledge we intend to use in development of our solution.
            Additionally, build intuition (\textit{intuition}) on how your solution was inspired.
            Finally, (\textit{solution}) state the proposed solution and its associated theorems.
            \alert{All proofs are listed in appendix}.
        \eo{itemize}
        \bitem{Pure Math Path:}
        \bo{itemize}{0.125cm}
            \item Set up (\textit{set up}) the mathematical prerequisite knowledge we intend to use in development of our solution.
            Finally, build intuition (\textit{intuition}) on how your solution was inspired.
            \alert{All proofs are listed in appendix}.
        \eo{itemize}
        \bitem{Experimental Path:}
        \bo{itemize}{0.125cm}
            \item Set up (\textit{set up}) the appropriate prerequisite knowledge required in the discussed experimental set up.
            Finally, build intuition (\textit{intuition}) behind the intended experimental set up.
        \eo{itemize}
    \ei{enumerate}
}

\bettersection{Results}

\betterframe{Results Slide}{
    \bo{itemize}{0.25cm}
        \bitem{Applied Math Path:}
        \bo{itemize}{0.125cm}
            \item Discuss the real-world and/or numerical experiment (\textit{experiment}) you intend to use to validate your solution.
            Include what sampling methods (\textit{methods}) were used and the necessary information to recreate the results. 
            Finally, discuss qualitative and quantitative results that best describe, and argue for, the solution methods existence in literature.
        \eo{itemize}
        \bitem{Pure Math Path:}
        \bo{itemize}{0.125cm}
            \item Provide a detailed sketch (\textit{un-detailed sketch}) a \alert{un-detailed} sketch of the mathematical problem's proof.
            Finally, develop more \alert{detailed} slides  (\textit{detailed sketch}) the align with, are in sequence with, the aforementioned proof sketch.
            \alert{All proofs are listed in appendix}.
        \eo{itemize}
        \bitem{Experimental Path:}
        \bo{itemize}{0.125cm}
            \item Provide a detailed analysis (\textit{results}) of statistically relevant results gathered from the experiment conducted.
            Finally, (\textit{interpretation}) argumentatively discuss your understanding of the results and/or develop a collection of probable / interesting hypothesis.
        \eo{itemize}
    \eo{itemize}
}

\bettersection{Conclusions and Future Work}

\betterframe{Conclusions and Future Work Slide}{
    \bo{itemize}{0.25cm}
        \bitem{Applied Math Path:}
        \bo{itemize}{0.125cm}
            \item Restate (\textit{problem focus}) your problem focus in one sentence.
            Review (\textit{solution}) in one sentence how your solution method was designed.
            Provide (\textit{positive conclusion}) a take home message on why your method achieves that current mathematical tools can not.
            Finally, (\textit{future work}) mention how your solution could be continued and/or applied.
        \eo{itemize}
        \bitem{Pure Math Path:}
        \bo{itemize}{0.125cm}
            \item Restate (\textit{problem focus}) your problem focus in one sentence.
            Review (\textit{solution}) in one sentence how your solution was developed.
            Provide (\textit{positive conclusion}) a take home message on how you results might expand human understanding.
            Finally, (\textit{future work}) mention how your solution could be continued and/or applied.
        \eo{itemize}
        \bitem{Experimental Path:}
        \bo{itemize}{0.125cm}
            \item See Pure Math Path.
        \eo{itemize}
    \eo{itemize}
}

\end{document}